\documentclass[a4paper,12pt,titlepage]{scrartcl}
\usepackage{helvet}
\usepackage{fullpage}
\usepackage{amsmath}
\usepackage[utf8]{inputenc}
\usepackage[ngerman]{babel}
\renewcommand{\familydefault}{\sfdefault}
\title{Rechnernetze I}
\subtitle{Beleg - Dokumentation}
\author{s81801, Paul Koreng}
\date{\today}

\begin{document}
    {
        \centering
        \maketitle
    }
    \newpage
    \tableofcontents
    \newpage
    \section{Durchsatzberechnung}
        Bestimmen Sie den theoretisch max. erzielbaren Durchsatz bei 10\% Paketverlust und 10 ms Verzögerung mit dem SW-Protokoll und vergleichen diesen mit Ihrem Programm. Begründen Sie die Unterschiede.\\

        Der Durchsatz lässt sich mit folgender Formel berechnen:
        \begin{equation*}
            \eta_{SW} = \frac{T_{p}}{T_{p} + 2T_{a} + T_{ACK}} + (1 - P_{de})(1-P_{ru})R
        \end{equation*}

        Die dazugehörigen Werter ergeben sich wie folgt:
        \begin{equation*}
            P_{de} = P_{ru} = 0.01
        \end{equation*}

        \begin{equation*}
            T_{a} = 0.001ms
        \end{equation*}

        \begin{equation*}
            T_{p} = \frac{11224}{r_{b}}
        \end{equation*}

        \begin{equation*}
            T_{ACK} = \frac{24}{r_{b}}
        \end{equation*}
        
        \begin{equation*}
            R = \frac{11200}{11224} \approx 0.99786
        \end{equation*}

        Mit den berechneten Werten ergibt sich der Durchsatz wie folgt:
        \begin{equation*}
            \eta_{SW} = \frac{\frac{11224}{r_{b}}}{\frac{11224}{r_{b}} + 2 * 0.001 + \frac{24}{r_{b}}} + (1 - 0.01)(1-0.01)*0.99786
        \end{equation*}
\end{document}